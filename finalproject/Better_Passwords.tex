\documentclass{article}
\usepackage{color}
\usepackage{cite}
\usepackage{hyperref}
\usepackage{graphicx}
\newcommand{\hl}[1]
{\colorbox{yellow}{#1}}
\title{Making Secure Easy-to-Remember Passwords}
\author{Philip Braunstein}
\date{December 14, 2015}
\begin{document}

\maketitle
\abstract{
Three password-making strategies are evaluated in this report: random strings of characters, numbers, and symbols; abbreviations of phrases mixed with meaningful numbers; and an adjective-noun-verb-adjective-noun string modeled after xkcd 936 \hl{SOURCE: XKCD}. Passwords are evaluated on ease of memorization, cryptographic strength against a password cracker, and bits of entropy
}

\section*{Introduction}
People like to imagine that computer security is usually compromised by genius hackers exploiting inscrutable vulnerabilities. Often however, an attacker compromises a system because of seemingly stupid reasons like the being written on a note next to the computer. Shoring up technical security is a worthy cause, but this effort is rendered irrelevant unless the impact of social engineering resulting in password leaks is minimized. 

People avoid changing their passwords, leave them written in plain text in a document on their computer or even on a sticky note on their computer for one reason only: secure passwords are hard to remember. Insecure and poorly-stored passwords are the cause of many security leaks \hl{FIND SOME GENERAL SOURCES}. Passwords are a common source of vulnerability because different websites have different requirements for what they consider valid passwords, and passwords that are considered secure are obtuse and hard for people to remember.

In this report, three methods for making passwords are described and evaluated. The security and how easy each type of password to remember is evaluated and described in the Results section.

\section*{To the Community}

\section*{Methods}
\subsection*{Password Generation Schemes}
\subsubsection*{Random Strings}
A random string password consists of ten random characters, numbers, and symbols. A random string passwords must have contain at least one of three of the four categories: upper-case letter, lower-case letter, number, or symbol. The script \texttt{randomPass.py} generates ten of these random strings passwords. Test subjects select one of the ten passwords to use.


\subsubsection*{Memorable Phrase with Memorable Number}
\subsubsection*{Modified XKCD Method}

\section*{Results}

\section*{Applications}

\section*{Conclusion}


\end{document}